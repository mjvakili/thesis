\newcommand{\todo}[1]{{\em \textcolor{red}{ #1}}}
\newcommand{\beq}{\begin{equation}}
\newcommand{\eeq}{\end{equation}}
\newcommand{\lang}{\langle}
\newcommand{\ra}{\rangle}
\newcommand{\vep}{\bm{\epsilon}}
\newcommand{\ep}{\epsilon}
\newcommand{\pars}{\vec{\theta}}
\newcommand{\dev}{\mathrm{d}}
\newcommand{\mstar}{h^{-1}M_\odot}
\newcommand{\hst}{\project{HST}}
\newcommand{\wfc}{\project{WFC3}}

\chapter{Super-resolution PSF model of HST WFC3-IR\chaplabel{wfc3ir}}

This \paper is joint work with Ross~Fadely (Insight) and David~W.~Hogg (NYU) and it is being prepared for submission. 

\section{Chapter abstract}

Accurate model of the Point Spread Function is crucial for reliable point source photometry, astrometry, and weak lensing studies.
The PSF model of the $\hst$ $\wfc$ IR channel does not meet the accuracy required by these science goals. In addition, the PSF of the $\hst$ $\wfc$ IR 
channel is poorly sampled. Weak lensing studies of poorly resolved images of faint distant galaxies demands 
a great knowledge of the PSF sampled at a resolution higher than of the $\hst$ $\wfc$ pixels.
In this investigation, we present a generative model of every image taken by the $\wfc$ IR channel as a set of a number of point sources convolved
with the instrument PSF. In particular, we focus on modeling the pixel-convolved PSF, the instrument PSF convolved with the pixel response function. 
In particular, we model the images of point sources observed in the calibration program of the $\hst$ $\wfc$ IR channel in the F160W bandpass. We 
expect the inference of the super-resolution PSF in the other bandpasses of the IR channel to follow a simillar procedure. 
We find that we can find th optimal soution of the problem by using a variance model that correctly takes into account the model 
uncertainty and a regularization term that imposes a smoothness condition on the super-resolution PSF 

\section{Introduction}

The Point Spread Function (hereafter PSF) determines the fraction of photons from a given point source that lands on a particular location 
from the center of that point source on a detector. The PSF of the $\hst$ $\wfc$ camera is extremely undersampled. That is, if a center of an observed star lies 
on the center of a pixel, a significant fraction of the brightness of that star will be 
encapsulated by the detector's pixel that contains the centroid of the star.
In other words, the full-width half-maximum (FWHM) of the PSF is not spanned by multiple pixels since the designed 
pixels are large. This is mainly the result of a compromise made in design of the detectors to cover a wider field of view.

Poor sampling of the PSF by detectors renders many astronomical image processing tasks difficult. Precise astrometry and photometry 
of individual point sources requires knowledge of the PSF sampled on higher resolution than that of the detector pixels. 
In uncalibrated observations, the images of stars are generated by convolution of the point sources with the PSF and multiplication 
with a non-uniform detector sensitivity called the \emph{flat-field}. Flat-field corrections are important for photometry and astrometry of 
individual point sources. In under-sampled images, capturing the sub-pixel variations of the flat-field requires having a higher resolution 
model of the PSF. 

Furthermore, cosmic shear studies require accurate measurement of the shapes of individual distant galaxies. The galaxies used for estimation 
of the cosmic shear signal are faint and barely resolved by the $\hst$ $\wfc$ detectors. In order to reliably estimate the ellipticities of these poorly 
resolved galaxies through model fitting, we need to convolve the model describing the light distribution of these galaxies with a higher resolution PSF model. 
Therfore, having access to a higher resolution model of the PSF is an essential ingredient in weak lensing studies of galaxies detected by the $\hst$ $\wfc$ IR 
channel.   

In this investigation, we focus our attention to modeling the pixel-convolved PSF which is the optical (instrumental) convolved with the 
pixel response function. An important step in all weak gravitational lensing studies is fitting a model to this pixel-convolved PSF which then will be used in making accurate estimates of the galaxy ellipticities and eventually the cosmic shear signal. 
From now on, we refer to the pixel-convolved PSF as the PSF. 

%A significant body of works are focused on developing a physical model of the optical PSF. 

The factor by which the PSF is undersampled varies from one filter to another.
The \wfc camera consists of six filters: F105W, F125W, F140W, F160W. In what follows in the rest of this chapter, we focus on modelling the super-resolution PSF in the FLT frame of F160W filter. The data is collected from targeting the globular cluster Omega Centauri during the calibration program.

Our strategy for inferring the super-resolution of the PSF model of F160W filter is as follows. 
First we run the Source Extractor algorithm to identify all the stellar sources in the observed data. 

In patch $n$, the model is given by

\begin{eqnarray}
\mathbf{y}_{n} &=&  \mathbf{m}_{n} + \mathrm{noise} \\
m_{n,i} &=& f_{n}K^{(n)}_{il} (\Delta_n) X_{l} + b_{n} 
\end{eqnarray} 
where the variance of the noise model adopted in this work is the following:
\beq
s_n^2 = \sigma_{n}^{2} + g.m_{n}
\eeq    

Ideally one would use the bright isolated sources to estimate the PSF. A considerable fraction of the observed stars are present in the crowded fields. 
In order to alleviate the issue arising from the contamination of stellar patches by the light coming from overlapping sources, we add another term to the noise model which accounts for the discrepancy between the downsampled PSF model on the data grid and the observed star. 

\beq
s_n^2 = \sigma_{n}^{2} + g.m_{n} + qm_{n}^{2}
\eeq 


The log-likelihood function for each patch is written as follows. 
\beq
-2 \ln L_{n} = \sum_{i=1}^{N_{\rm pix}} \frac{\big(y_{i} - m_{i} \big)^{2}}{s^{2}_{n}} + \ln(s_{n}^{2})
\eeq

Our strategy for finding the optimal values of the parameters of individual point sources $\{f_{n},b_{n},\Delta_{n}\}_{n=1}^{N}$ and the global solution of the
super-resolution PSF $X$ is as follows:

\begin{algorithm} 
\caption{The procedure for Inferring the Super-Resolution PSF}
\begin{algorithmic}[1] \label{alg:abcpmc}
%\STATE \DATA: D
%\STATE \RESULT: ABC posterior sample of $\pars$
\IF{$t=1:$}
%\STATE $\epsilon_t \gets \infty$
\FOR{$n=1,...,N$}
   \STATE // \emph{This can now be done in parallel for all i}
   %\WHILE{$\rho(X,D)>\epsilon_t$}
   \STATE $b_{n}$ \gets \mathrm{median} \; $(y_{n})$
   \STATE $f_{n}$ \gets $\sum_{i=1}^{N_pix} \big(y_{n,i} - b_{n}\big)$
   \STATE \mathrm{Initialize} $\Delta_{n}$ with the 3\time3 polynomial method
   \STATE $X_{n}$ \gets \mathrm{cubic \; spline} $(\Delta_n)[\big(y_{n,i} - b_{n}\big)/f_n]$
\ENDFOR
\STATE $X$ \gets $\sum_{n=1}^{N} X_{n]$
   %\ENDWHILE
\ENDIF
\IF{$t=2,...,N_{it}:$}
\FOR{$i=1,...,N$}
   \STATE // \emph{This loop can now be done in parallel for all i}
   \WHILE{$\rho(X,D)>\epsilon_t$}
   \STATE Draw $\pars^{*}_{t}$ from $\{\pars_{t-1}\}$ with probabilities $\{w_{t-1}\}$
   \STATE $\pars^{*}_{t} \gets K(\pars^{*}_{t},.)$
   \STATE $X = f(\pars^{*}_{t})$
   \ENDWHILE
   \STATE $\pars^{(i)}_{t} \gets \pars^{*}_{t}$
   \STATE $w^{(i)}_{t} \gets \pi(\pars^{(i)}_{t}) / \big(\sum\limits_{j=1}^{N}w_{t-1}^{(i)}K(\pars^{(j)}_{t-1},\pars^{(i)}_{t}) \big)$
\ENDFOR
\ENDIF
\end{algorithmic}
\end{algorithm}



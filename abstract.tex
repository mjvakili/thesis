One of the essential steps in cosmological weak lensing analyses is accurate characterization of the the effect of telescope optics and atmosphere (in the case of ground-based imaging) on the light profiles of galaxies. This effect is captured by a kernel known as the Point Spread Function (PSF) that needs to be fully estimated and corrected for. We address two challenges a head of accurate PSF modeling for weak lensing studies. The first challenge is finding the centers of point sources that are used for estimating the PSF. We show that the approximate methods for centroiding stars in wide surveys are able to optimally saturate the information content that is retrievable from astronomical images in the presence of noise. Furthermore, we demonstrate how with the detectors of the space-based telescopes we can infer the PSF from the poorly sampled images. In particular, we present a 
forward model of the point sources as observed by the HST WFC3 IR channel. We show that we can accurately estimate the super-resolution PSF. We also introduce a noise model that permits us to robustly analyze the HST WFC3 IR observations of the crowded fields.    

Modeling of galaxy clustering on small scales requires assuming a halo model. Clustering of halos has been shown to depend on halo properties beyond mass such as halo concentration, a phenomenon referred to as assembly bias. Standard large-scale structure studies assume that halo mass alone is sufficient in characterizing the connection between galaxies and halos. However, modeling of galaxy clustering can face systematic effects if the expected number of galaxies in halos are correlated with other halo properties. Using high resolution N-body simulations and the clustering measurements of Sloan Digital Sky Survey DR7 main galaxy sample, we present the extent to which the dependence of galaxy clustering modeling on halo properties beyond mass can improve our modeling of galaxy clustering.

Measurements of galaxy clustering with the low-redshift galaxy surveys provide sensitive probe of cosmology and growth of structure. One of the key ingredients in precise parameter inference using galaxy clustering is accurate estimation of the covariance matrix of cosmological observables. This requires generation of many independent galaxy mock catalogs that accurately describe the statistical distribution of galaxies in a wide range of physical scales. We present a fast and accurate method based on low-resolution N-body simulations and automatic bias estimation for generating mock catalogs. We show that this approach is able to recover galaxy clustering at a percent level accuracy down to quasi-nonlinear scales. 

Cosmological datasets are interpreted by specifying a likelihood function that is often assumed to be multivariate Gaussian. Likelihood free approaches such as Approximate Bayesian Computation (ABC) can bypass this assumption by introducing a generative forward model of the data and a distance metric for quantifying the closeness of the data and the model. We present the first application of ABC in large scale structure for constraining the connections between galaxies and dark matter halos. We show that ABC equipped with importance sampling and a generative forward model of the data that incorporates sample variance and systematic uncertainties can accurately estimate the parameters given the galaxy clustering and galaxy group statistics measurements.  

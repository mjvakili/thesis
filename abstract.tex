Modeling of galaxy bias on small scales requires assuming a halo model. Clustering of halos has been shown to depend on halo properties beyond mass such as halo concentration, a phenomenon referred to as assembly bias. Standard large-scale structure studies assume that halo mass alone is sufficient in characterizing the connection between galaxies and halos. However, modeling of galaxy bias can face systematic effects if the number of galaxies are correlated with other halo properties. Using high resolution N-body simulation and the clustering measurements of Sloan Digital Sky Survey DR7 main galaxy sample, I present the extent to which the dependence of galaxy bias on halo properties beyond mass can improve our modeling of galaxy clustering.

Measurements of galaxy clustering with the low-redshift galaxy surveys provide sensitive probe of cosmology and growth of structure. One of the key ingredients in precise parameter inference using galaxy clustering is accurate estimation of the covariance matrix of cosmological observables. This requires generation of a large number of independent galaxy mock catalogs that accurately describe the statistical distribution of galaxies in a wide range of physical scales. I present a fast method based on low-resolution N-body simulations and automatic bias estimation for generating mock catalogs. 
I show that my method is able to recover galaxy clustering at a percent level accuracy down to quasi nonlinear scales. 


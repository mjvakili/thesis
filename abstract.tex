State of the inhomogeneous universe and its geometry throughout cosmic history can be studied by measuring the clustering of galaxies and the gravitational lensing of distant faint galaxies. Lensing and clustering measurements from large datasets provided by modern galaxy surveys will forever shape our understanding of the how the universe expands and how the structures grow. Interpretation of these rich datasets requires careful characterization of uncertainties at different stages of data analysis: estimation of the signal, estimation of the signal uncertainties, model predictions, and connecting the model to the signal through probabilistic means. In this thesis, we attempt to address some aspects of these challenges.  

The first step in cosmological weak lensing analyses is accurate estimation of the distortion of the the light profiles of galaxies by large scale structure. These small distortions, known as the cosmic shear signal, are dominated by extra distortions due to telescope optics and atmosphere (in the case of ground-based imaging). This effect is captured by a kernel known as the Point Spread Function (PSF) that needs to be fully estimated and corrected for. We address two challenges a head of accurate PSF modeling for weak lensing studies. The first challenge is finding the centers of point sources that are used for empirical estimation of the PSF. We show that the approximate methods for centroiding stars in wide surveys are able to optimally saturate the information content that is retrievable from astronomical images in the presence of noise. 

The fist step in weak lensing studies is estimating the shear signal by accurately measuring the shapes of galaxies. Galaxy shape measurement involves modeling the light profile of galaxies convolved with the light profile of the PSF. Detectors of many space-based telescopes such as the Hubble Space Telescope ($\textsl{HST}$) sample the the PSF with low resolution. Reliable weak lensing analysis of galaxies observed by the $\textsl{HST}$ camera requires knowledge of the PSF at a resolution higher than the pixel resolution of $\textsl{HST}$. This PSF is called the super-resolution PSF. In particular, we present a forward model of the point sources imaged through filters of the $\textsl{HST}$ $\textsl{WFC3}$ IR channel. We show that this forward model can accurately estimate the super-resolution PSF. We also introduce a noise model that permits us to robustly analyze the $\textsl{HST}$ $\textsl{WFC3}$ IR observations of the crowded fields.    

Then we try to address one of the theoretical uncertainties in modeling of galaxy clustering on small scales. Study of small scale clustering requires assuming a halo model. Clustering of halos has been shown to depend on halo properties beyond mass such as halo concentration, a phenomenon referred to as assembly bias. Standard large-scale structure studies with halo occupation distribution (HOD) assume that halo mass alone is sufficient to characterize the connection between galaxies and halos. However, assembly bias could cause the modeling of galaxy clustering to face systematic effects if the expected number of galaxies in halos is correlated with other halo properties. Using high resolution $N$-body simulations and the clustering measurements of Sloan Digital Sky Survey ($\textsl{SDSS}$) DR7 main galaxy sample, we show that modeling of galaxy clustering can \emph{slightly} improve if we allow the HOD model to depend on halo properties beyond mass.

One of the key ingredients in precise parameter inference using galaxy clustering is accurate estimation of the error covariance matrix of clustering measurements. This requires generation of many independent galaxy mock catalogs that accurately describe the statistical distribution of galaxies in a wide range of physical scales. We present a fast and accurate method based on low-resolution $N$-body simulations and an empirical bias model for for generating mock catalogs. We use fast particle mesh gravity solvers for generation of dark matter density field and we use Markov Chain Monti Carlo (MCMC) to estimate the bias model that connects dark matter to galaxies. We show that this approach enables the fast generation of mock catalogs that recover clustering at a percent-level accuracy down to quasi-nonlinear scales. 

Cosmological datasets are interpreted by specifying likelihood functions that are often assumed to be multivariate Gaussian. Likelihood free approaches such as Approximate Bayesian Computation (ABC) can bypass this assumption by introducing a generative forward model of the data and a distance metric for quantifying the closeness of the data and the model. We present the first application of ABC in large scale structure for constraining the connections between galaxies and dark matter halos. 
We present an implementation of ABC equipped with Population Monte Carlo and a generative forward model of the data that incorporates sample variance and systematic uncertainties.
We show that this framework permits us to accurately infer the galaxy/halo connection with the galaxy clustering and galaxy group statistics measurements.

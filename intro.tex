\chapter*{Introduction}\addcontentsline{toc}{chapter}{Introduction}

We are entering a new era of cosmology. Thanks to the ongoing and 
the upcoming galaxy surveys, we are going to be able to measure cosmological 
parameters within a few percent level precision. This ambitious goal 
can only be achieved if we keep the uncertainties arising from 
astrophysical and observation systematics below the statistical errors.
 
%precision cosmology. The ongoing and the upcoming galaxy surveys 
%are going to forever shape our understanding of the universe. 

%Furthermore, there is a wealth of information in small scale galaxy clustering. 
%We can place unprecedented constraints on the cosmological parameters and the growth of structure. 
Furthermore, rigorous probe of the clustering of galaxies on small scales 
can shed light on how galaxies form and how they trace the underlying dark matter density field.
Characterization of the galaxy clustering signal on small scales requires assuming a halo model. 
Combining multiple probes such as galaxy-galaxy auto correlation function and galaxy-matter cross 
correlation has been shown to provide a powerful cosmological probe after marginalizing out 
the galaxy-halo connection.

With the spectroscopic surveys such as eBOSS \citep{eboss}, DESI , and Euclid \citep{euclid}, 
we are going to precisely constrain the expansion history of the universe as well as the growth of structure. 
With these constraints, we will be able to distinguish different physical scenarios that explain the accelerating expansion of the universe. 
A crucial component of BAO and RSD analyses is estimation of the uncertainties of the statistical summaries of 
the data obtained from galaxy surveys. One of the most promising frameworks for estimating the uncertainties in form of covariance matrices is production 
of a large number of accurate simulations of our observations. 

Furthermore, we live in the era of systematic dominated measurements. 
Therefore it is important to make use of statistical techniques that mitigate 
biases arising from these systematic errors. It is important to design end-to-end simulations of the survey.

Weak gravitational lensing is the study of the distortion of lights coming from distant galaxies by 
the intervening large scale structure. Cosmic shear is a direct probe of the distribution of 
matter in the universe. Cosmic shear analysis in multiple tomographic bins is going to 
constrain the dark energy equation of state and growth of structure. 
Extraction of the cosmological information from the galaxy images is a procedure far from 
trivial. This is due to the fact that the cosmic shear signal is dominated by systematics such as the PSF. 


Galaxy surveys will provide a valuable legacy dataset that can be used for detailed galaxy formation studies. 
These investigations will provide a window into how the visible structures in the observable universe 
trace the underlying invisible dark matter density. Furthermore, more comprehensive knowledge of galaxy-dark matter connection 
can inform cosmological parameter inferences with galaxies. 



 




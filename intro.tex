\chapter*{Introduction}\addcontentsline{toc}{chapter}{Introduction}

Cosmology is entering a new era. Thanks to the ongoing and 
the upcoming low redshift galaxy surveys as well as the early universe probes, 
we are able to test cosmological theories with high preceision.
With the observations of the cosmic microwave background (CMB) radiation 
and the distant Type Ia supernovae, our understanding of universe was 
revolutionized in the late 90s. 

The early universe was in a hot and dense state where 
matter and radiation formed a primordial plasma. Eventually, the 
plasma cooled down, first atoms formed and photons started streaming freely.  
Today, we can observe these ancient photons as a microwave background radiation. 
The temperature of this radiation is approximately $2.7 \; K$ in all directions with 
fluctuations that are 1 part in 10000.

In a strikingly great agreement with the CMB observations, models of inflation---nearly 
exponential rapid phase in the expansion of early universe--- 
predict \emph{nearly Gaussian and scale invariant} random fluctuations around a homogeneous 
background. Microscopic quantum fluctuations generated in an inflationary 
stage were stretched to cosmic volumes. Evolution of these initial seeds 
resulted in the formation of structures such as planets, stars, galaxies, clusters of galaxies, filaments, \etc\.

By measuring the brightness of supernovae type Ia, we have also learned 
that supernovae Ia are fainter than what we expect them to be in an 
expanding universe filled with matter. This surprising observation led us to 
believe that approximately 70 $\%$ of the energy budget of the universe is 
given by a dark energy component which can be though of as liquid with negative pressure.

From a theoretical standpoint, cosmic acceleration can be explained in numerous modified gravity frameworks. 
Despite the astonishing success of general relativity in certain regimes such as 
the solar system, binary pulsars, and gravitational waves as a result of merging supermassive blackholes, 
precision test of GR on large scales remain an active area of research. Current constraints on 
the accelerated expansion and growth of structure can not rule out some theories of 
modified gravity.

Detection of the Baryonic Acoustic feature in clustering of galaxies further enhanced 
our understanding of the expansion history. Acoustic oscillations in the early universe plasma 
result in a sound wave that leaves an imprint with a characteristic scale on the perturbations 
in late time universe. Therefore the BAO provides a highly accurate estimate of the angular diameter 
distance and it provides a powerful probe of the geometry and expansion history of the universe. 

In galaxy redshift surveys, the distances to galaxies can be measured by the measured 
redshifts. The distance information provided by these surveys is distorted by the peculiar 
velocities of galaxies along the line of sight. These distortions are referred to as redshift space 
distortions (RSD). Redshift space distortions are highly sensitive probes of the growth of structure 
and contain additional cosmological information than what can be provided by BAO. Thus measurements of 
galaxy clustering with redshift surveys can constrain dark energy and laws of gravity.

With the spectroscopic surveys such as eBOSS \citep{eboss}, DESI, and Euclid \citep{euclid}, 
we are going to precisely constrain the expansion history of the universe as well as the growth of structure. 
With these constraints, we will be able to distinguish different physical scenarios that explain the accelerating expansion of the universe. 
A crucial component of BAO and RSD analyses is estimation of the uncertainties of the statistical summaries of 
the data obtained from galaxy surveys. One of the most promising frameworks for estimating the uncertainties in form of covariance matrices is production 
of a large number of accurate simulations of our observations. 

Gravitational lensing of luminous sources (galaxies, CMB) can measure 
the state of the inhomogeneous matter density and the time-dependence of dark energy. 
Lensing of the light emitted by background faint galaxies by the intervening large scale structure 
results in very small but correlated distortions in the shapes of galaxies. This phenomenon is referred to 
as cosmic shear.

Cosmic shear studies are hurdled by a number of statistical uncertainties (unknown distribution 
of galaxy intrinsic morphologies) and systematics. The long list of systematic error budget 
of cosmic shear studies can in general fall into the following categories: uncertain distance 
(photometric redshift) information, and imperfect knowledge of intrinsic alignments of galaxies and non-linear matter 
power spectrum models, and image systematics. 

Tomographic reconstruction of the distribution of dark matter with imaging surveys requires 
accurate distance information which is encoded in the redshifts of galaxies. 
The redshift estimated from the imaging surveys is usually uncertain as it is measured with coarse 
spectra in the form of a few magnitude numbers based on broadband photometry. These redshifts are called 
photometric redshifts and their accuracy depends on the number of broadband filters and the overlapping spectroscopic 
sample of galaxies \citep{bonnett2016,choi2016,boris2016,hildebrandt2017}. %FIXME BORIS AND HOGG?

Intrinsic alignment of galaxies (IAs) also contaminate the cosmic shear signal. These IAs, if not accounted for, 
can significantly bias the cosmological parameter inference \citep{codis2015,joachimi2015, kirk2015,krause_ia}. 
Cosmological interpretation of the cosmic shear signal requires an accurate knowledge of the nonlinear 
matter power spectrum \citep{semboloni2013, eifler2015, schaye2015, joudaki2016, kitching2016, mead2016}. Modeling the role of 
Baryonic feedback and the accuracy of the emulators for nonlinear matter clustering remains an active area of research. 

Perhaps the most widely and extensively studied sources of systematics in weak lensing are the image systematics which make 
the process of inferring the cosmic shear from noisy images of galaxies far from trivial. The light profile of 
galaxies is distorted with the atmosphere and telescope optics. The amount of distortion is captured in a kernel called 
the Point Spread Function (PSF). The size and anisotropy of the background galaxies used in weak lensing studies is 
smaller than those of the PSF, making the cosmic shear signal dominated by the PSF which is empirically estimated at 
the positions of stars and then interpolated to the positions of galaxies.  

Furthermore, additional systematics arise from unrealistic models of galaxy light profile \citep{voight2010,im3shape,kac2014}, 
detector effects \citep{arun2016,jaya2016,plazas2016}, nonlinear dependence of galaxy ellipticities 
on the pixel data in the presence of noise that biases the shear \emph{point} \emph{estimates} \citep{melchoir,great3,conti2017}, 
and incorrect model of the PSF or inaccurate PSF interpolation \citep{rowe2010,kuijken2015,great3,des}.

These systematics lead to shear calibration biases in the form of additive and multiplicative 
shear biases. For instance, an incorrect size or ellipticity in the PSF model could introduce 
a multiplicative bias shear. This bias gives rise to a spurious correlation function with the 
same order of magnitude as the cosmological signal.

The shear biases are either calibrated with realistic image simulations \citep{im3shape,jee2016,conti2017} or through cross correlation of 
the cosmic shear signal with other cosmic probes that do not suffer from the same biases (\eg\ galaxy positions)
\citep{liu2016,schann,singh2017}. 

Simulation based techniques are limited due to difficulty in 
generation of realistic galaxy image simulations \citep{great3,lanus2017} and our limited ability to match the depth and detection limits of 
the imaging surveys \citep{hoekstra}. The limitation of cross-correlation technique is that it makes strong assumptions about the assigning a 
single bias parameter to a large population of galaxies under consideration. The shear biases have been shown to be scale-dependent \citep{des,jee2016} 
and depend on galaxy properties \citep{im3shape,conti2017}. 

Novel shear inference models that do not rely on delivering a point-estimate for the shear signal \citep{schneider,bernstein,huff,sheldon} 
provide promising venues for mitigation of biases but their accuracy is limited to the accuracy in the PSF model.
In chapters 1 and 5, we try to address two challenges in accurate PSF modeling that could lead to biases in cosmological weak lensing analysis. 

One of the steps of astronomical image processing that could lead to inaccurate model of the PSF is centroiding 
of stars. An erroneous method for determination of the centroid of stars could lead to both inaccurate estimation of the 
PSF at the positions of stars and inaccurate PSF interpolation \citep{anderson2000,lupton,anderson2003,desc}. Large astronomical surveys rely on approximate 
methods for centroiding the stars \citep{sextractor,sdss,des}. 

In chapter one we argue that in the presence of noise, there exist
a lower bound on the error arising from centroiding methods. We show that this theoretically set lower bound, also known as the Cram\'{e}r-Rao lower bound \citep{lecam}, 
is \emph{almost} saturated by centroiding methods that rely on correlating the images of stars with the PSF or some approximation to the 
PSF. These centroiding methods are called matched-filter centroiding methods and in certain limits provide an approximation to the more accurate 
methods that rely on fitting a model to the stars. 

Additionally, weak lensing shear couples the short wavelength and long wavelength modes of the galaxy light profiles. 
Therefore estimation of the response of galaxy light profile to shear requires knowing the model of galaxy image 
with a resolution better than the PSF that the image of galaxy is convolved with. 

In most space telescope \citep{euclid,wfirst} the detectors are designed to be large in order to yield wider field of view. 
That is, the detectors sample the light profile of the PSF at a lower resolution. In such cases, a large fraction of the light of a given point source 
is captured by the pixel that contains the centroid of the point source. In other words, the PSF is \emph{undersampled}.

Most galaxies of interest for weak lensing studies with these space-based weak lensing experiments are unresolved. 
As a consequence, estimating the shear from unresolved galaxy images requires 
knowing the the galaxy light profile model and the PSF at higher resolutions than the native pixel size of the telescopes \citep{olic,ngole2}.

The filters in HST WFC3 IR channel have the most undersampled detectors among all HST filters. 
As such, HST WFC3 IR channel could benefit a lot from an accurate model of its super resolution PSF. 
In this chapter 5, we present the first generative forward model of stellar sources as observed by the HST camera.
Unlike previous attempts at inferring the PSF model of HST WFC3, our model is data driven in that it makes 
use of the observed data and it presents a proper treatment of modeling in the presence of damaged pixels, overlapping 
point sources, \etc\.  

Galaxies are luminous tracers of the large scale structure. Thus study of galaxy clustering as a means to understand the growth of structure is 
one of the key drivers of low redshift galaxy surveys. Significant progress has been made in linking the observations of 
galaxies and our theoretical understanding of the nonlinear evolution of dark matter. 

On large scales, cosmological perturbation theory equipped with a prescription for galaxy bias permits us 
to accurately model the clustering of galaxies. However, nonlinear evolution of matter 
poses a challenge to applicability of perturbative approaches in small scales. In such limits, 
galaxy clustering must be understood within the context of the halo model which has had 
much success in describing time evolution of the clustering of matter.  

Modeling galaxy clustering with the halo model follows two central assumptions. First, galaxies form in virialized regions of 
dark matter over density known as halos. Clustering of galaxies is governed by clustering of dark matter halos. 
In order to use galaxy clustering measurements for gaining insight into cosmological structure formation models we need to 
know how clustering of galaxies can be determined from clustering of halos. That is, we are required to specify how halos 
are populated with galaxies \citep{seljak2000,scoccimarro2001,berlind_weinberg2002}.

One of the successful prescriptions for assignment of galaxies to halos is the halo occupation distribution (HOD). 
HOD provides a prescription for the expected number of galaxies that reside a halo as well as 
the positions and velocities of galaxies that are distributed within a halo. This simple assumption has been useful 
in reproducing the observed statistics of galaxies \citep{tinker_rsd2007,zehavi2011,zheng_guo}.

Despite being successful at reproducing a wide range of the observed statistics of galaxies, the simple \emph{mass-only} HOD 
remains challenged by a theoretical phenomenon known as assembly bias. Assembly bias, seen in $N-$body simulations, states that at a fixed halo mass, clustering 
of halos depend on halo properties beyond mass such as their formation history, gravitational potential, \etc\ \citep{weschler2006,gao2007,arz2014,sunayama2016}.
This theoretical prediction has not yet been confirmed with observations and there are very mixed results in the literature.

By weak lensing analysis of the galaxy groups in GAMA (GAlaxy and Mass Assembly, see \citealt{driver}) 
with the KiDS (Kilo Degree Survey, see \citealt{kuijken2015}) imaging data, \citet{dvornik2017} find no evidence for halo assembly bias.
Using the galaxy-galaxy lensing and clustering measurements of SDSS galaxy \emph{redmapper} clusters \citep{rykoff} with halo masses $10^{14}$ $M_{\odot}$, 
\citet{miyatake2016} claimed strong difference between the bias of two populations of clusters with different radial distribution of satellite galaxies 
(a proxy for the formation history of a dark matter halo). \citet{zuetal2016} argues that the findings of \citet{miyatake2016} is due to projection effects (as a result of highly uncertain photometric redshifts) and they cannot be interpreted as detection of assembly bias. 

The effect of assembly bias on galaxy clustering can be seen in many subhalo abundance matching (SHAM, see \citealt{hearin2014,lehman2015} and references therein) methods. SHAM assumes a one-to-one relation between halo (including subhalos) properties (\eg\ mass, circular velocity) and some galaxy properties (\eg\ stellar mass, luminosity). In Chapter 3, with the clustering measurements of SDSS DR7 main sample of galaxies \citep{abazajian2009} and the Small MultiDark $N$-body simulation, we show that the clustering predictions of \citep{decorated} HOD model that takes assembly bias into account are consistent with the predictions of the \emph{mass-only} HOD model within 1-$\sigma$ level. 

We also note that for the sample of $L_{\star}$ galaxies, there is \emph{slight} improvements in galaxy clustering predictions of the HOD model with assembly bias 
on large scales. Furthermore, we note that in terms of information criteria, there is no improvement when a more complex HOD model is used to fit the clustering measurements.
We note that in terms of the effect of halo assembly bias on galaxy clustering, our constraints show qualitatively similar behavior in comparison with the predictions of subhalo abundance matching models. Unlike the findings of \citet{zentner2016}, we do not find any evidence for assembly bias in the satellite population of galaxies.

One of the perplexities faced by contemporary cosmology is the discordance between the constraints on cosmological parameters from some of the low redshift probes and 
the Planck CMB results \citep{planckII}. This includes the cosmological constraints from the cosmic shear analysis of CFHTLenS \citep{heymans,kitching2016} and KiDs \citep{hildebrandt2017}. The disagreement appears in the constraints over the parameters $\sigma_{8}$ and $\Omega_{m}$, the amplitude of the linear matter power spectrum and the matter density parameters respectively. This tension exists at a 2-$\sigma$ level. 

The \emph{cosmic} \emph{discordance} persists when the low redshift probe under consideration is galaxy-galaxy lensing which contains information regarding galaxy-matter cross correlation. Similar to galaxy clustering, modeling the small scale galaxy-galaxy lensing signal requires assuming a halo model. 
This signal is an estimate of the lensing of the background galaxies from a \emph{deep} imaging survey and the foreground galaxies in a \emph{shallow} spectroscopic survey.

The galaxy-galaxy lensing measurements of SDSS III/BOSS \citep{miyatake15,lensingislow} are not consistent with the predictions of best-fit Planck cosmology \citep{planckII}.
The $\sigma_{8}-\Omega_{m}$ constraint from clustering and galaxy-galaxy lensing measurements of BOSS galaxies \citep{more15} does not match the Planck constraint.
Disagreements hold for the cosmological constraints---assuming a simple mass only HOD model---from the clustering and galaxy-galaxy lensing measurements of SDSS DR7 main sample \citep{cacciato13}. These discrepancies could signal a new physics, observational systematics, or uncertainties arising from the galaxy formation physics, including \emph{assembly} \emph{bias}.

Combining the early universe probes of cosmology and large scale structure requires accurate 
characterization of systematics and nuisance parameters in both datasets. In the context of large scale 
structure, models governing the galaxy-halo connection serve as nuisance parameters that we marginalize over. A possible source of uncertainty in characterization of galaxy-halo connection is assembly bias. In order to distinguish between physical scenarios pointing at potentially new physics and systematics, it is important to study the effect of halo assembly bias on low redshift cosmological probes including galaxy clustering and galaxy-galaxy lensing. In Chapter 3, we take a step toward better understanding the impact of assembly bias. 

The future low redshift probes and the key CMB datasets will enable us to map the 
density field across a considerable fraction of the observable universe. They will also provide us with 
a consistent picture of the initial conditions, formation of structure, and laws of gravity. 
Reaching a percent-level constraint over the cosmological parameters will 
forever shape our understanding of the cosmos. 

These ambitious goal however, can only be achieved if we keep the 
uncertainties arising from astrophysical and observation systematics 
below the statistical errors. In this dissertation, we attempt to take advantage of 
computational methods in order to address some sources of uncertainty that stand on the way of 
reaching precision cosmology. 

We live in the era of systematic dominated cosmological measurements. 
Therefore it is important to make use of statistical techniques that mitigate 
biases arising from these systematic errors. It is important to design end-to-end simulations of the survey.

%Galaxy surveys will provide a valuable legacy dataset that can be used for detailed galaxy formation studies. 
%These investigations will provide a window into how the visible structures in the observable universe 
%trace the underlying invisible dark matter density. Furthermore, more comprehensive knowledge of galaxy-dark matter connection 
%can inform cosmological parameter inferences with galaxies. 

In Chapters 2,3, and 4 we used the suite of MultiDark cosmological $N$-body simulations made publicly available in the CosmoSim database \footnote{\url{https://www.cosmosim.org}}. The CosmoSim database used in this paper is a service by the Leibniz-Institute for Astrophysics Potsdam (AIP). sThe MultiDark database was developed in cooperation with the Spanish MultiDark Consolider Project CSD2009-00064.

In Chapter 3, we used the measurements done with the SDSS DR7 \footnote{\url{http://classic.sdss.org/dr7/}} data \citep{abazajian2009}. 
Funding for the SDSS and SDSS-II has been provided by the Alfred P. Sloan Foundation, the Participating Institutions, the National Science Foundation, the U.S. Department of Energy, the National Aeronautics and Space Administration, the Japanese Monbukagakusho, the Max Planck Society, and the Higher Education Funding Council for England. The SDSS Web Site is http://www.sdss.org/. The SDSS is managed by the Astrophysical Research Consortium for the Participating Institutions. The Participating Institutions are the American Museum of Natural History, Astrophysical Institute Potsdam, University of Basel, University of Cambridge, Case Western Reserve University, University of Chicago, Drexel University, Fermilab, the Institute for Advanced Study, the Japan Participation Group, Johns Hopkins University, the Joint Institute for Nuclear Astrophysics, the Kavli Institute for Particle Astrophysics and Cosmology, the Korean Scientist Group, the Chinese Academy of Sciences (LAMOST), Los Alamos National Laboratory, the Max-Planck-Institute for Astronomy (MPIA), the Max-Planck-Institute for Astrophysics (MPA), New Mexico State University, Ohio State University, University of Pittsburgh, University of Portsmouth, Princeton University, the United States Naval Observatory, and the University of Washington.

In Chapter 5, all of the HST archival data were obtained from the Mikulski Archive for Space Telescopes (MAST) \footnote{\url{https://archive.stsci.edu}}. STScI is operated by the Association of Universities for Research in Astronomy, Inc., under NASA contract NAS5-26555.
 
Bulk of the computations in this work were carried out on the Mercer cluster which is part of the New York University High Performance Computing facilities \footnote{https://wikis.nyu.edu/display/NYUHPC/Clusters+-+Mercer}}. For a fraction of computations performed for this thesis, I acknowledge the use of machines in the Center for Cosmology and Particle Physics \footnote{http://ccpp.nyu.edu}}.
I also acknowledge the use of the computational resources in MareNostrum Cosmological Project \footnote{http://astro.ft.uam.es/marenostrum/}}. 



\chapter*{Introduction}\addcontentsline{toc}{chapter}{Introduction}

Cosmology is entering a new era. Thanks to the ongoing and 
the upcoming low redshift galaxy surveys as well as the early universe probes, 
we are able to put cosmological theories into test with high precision.
With the observations of the cosmic microwave background (CMB) radiation 
and the distant Type Ia supernovae, our understanding of universe was 
revolutionized in the late 90s. Later datasets advanced our understanding 
of the universe even further.

The early universe was in a hot and dense state where 
matter and radiation formed a primordial plasma. Eventually, the 
plasma cooled down, first atoms formed and photons started streaming freely.  
Today, we can observe these ancient photons as a microwave background radiation. 
The temperature of this radiation is approximately $2.7 \; K$ in all directions with 
fluctuations that are 1 part in 10000.

In a strikingly great agreement with the CMB observations, models of inflation---nearly 
exponential rapid phase in the expansion of early universe--- 
predict \emph{nearly Gaussian and scale invariant} random fluctuations around a homogeneous 
background. Microscopic quantum fluctuations generated in an inflationary 
stage were stretched to cosmic volumes. Evolution of these initial seeds 
resulted in the formation of structures such as planets, stars, galaxies, 
clusters of galaxies, filaments, \emph{etc}.

By measuring the brightness of supernovae type Ia, we have also learned 
that supernovae Ia are fainter than what we expect them to be in an 
expanding universe filled with matter. This surprising observation led us to 
believe that approximately 70 $\%$ of the energy budget of the universe is 
given by a dark energy component which can be considered a liquid with 
negative pressure.

From a theoretical standpoint, cosmic acceleration 
can be explained in numerous modified gravity frameworks. 
Despite the astonishing success of general relativity in 
certain regimes such as the solar system, binary pulsars, 
and gravitational waves as a result of merging supermassive blackholes, 
precision test of GR on large scales remains an active area of research. 
Current constraints on the accelerated expansion and growth of structure 
can not rule out certain theories of modified gravity.

Detection of the Baryonic Acoustic feature in clustering of galaxies 
further enhanced our understanding of the expansion history. 
Acoustic oscillations in the early universe plasma result in a sound wave 
that leaves an imprint with a characteristic scale on the perturbations 
in late time universe. Therefore the BAO provides a highly accurate estimate of the angular diameter 
distance and it provides a powerful probe of the geometry and expansion history of the universe. 

In galaxy redshift surveys, the distances to galaxies can be measured by the measured 
redshifts. The distance information provided by these surveys is distorted by the peculiar 
velocities of galaxies along the line of sight. These distortions are referred to as redshift space 
distortions (RSD). Redshift space distortions are highly sensitive probes of the growth of structure 
and contain additional cosmological information than what can be provided by BAO. Thus measurements of 
galaxy clustering with redshift surveys can constrain dark energy and laws of gravity.

%Spectroscopic surveys such as eBOSS \citep{eboss}, DESI, and Euclid \citep{euclid} allow us to precisely 
%constrain the expansion history of the universe as well as the growth of structure. 
%However, interpretation of these rich datasets requires characterization of the uncertainties on the clustering 
%measurements of these datasets. The uncertainties are estimated in terms of the covariance matrix. Precision matrix, 
%the inverse of the covariance matrix appears in the cosmological likelihood 
%(the probability of the observed data given the model).

%These constraints will enable us to distinguish between different physical scenarios that explain the accelerated expansion of the universe. 
%A crucial component of BAO and RSD analyses is estimation of the uncertainties of the statistical summaries of 
%the data obtained from galaxy surveys. One of the most promising frameworks for estimating the uncertainties in form of covariance matrices is production 
%of a large number of accurate simulations of our observations. 

The large datasets from galaxy surveys are analyzed by estimating the statistical summaries of the 
data and then interpreting them with the predictions from cosmological theories. This is done by writing down 
a likelihood function which is the probability of the observed data given the theoretical model.
Cosmological likelihood functions are often assumed to have a Gaussian functional form. Given this assumption, the likelihood 
can be fully specified with three ingredients: a data vector which is estimated from observations, a mean model vector 
which is computed from the predictions of theoretical models, and an inverse covariance matrix. 

Therefore, a crucial step in interpreting the data is computing the error covariance matrix. 
More importantly we need to know the inverse covariance matrix (precision matrix) precisely because this 
is the matrix that appears in the likelihood function. Modern cosmological analyses rely on computing the covariance matrix 
analytically or estimation of the matrix from a large suite of simulations or from internal resampling of the data. 

Motivated by the idea of minimizing the effect of the noise properties of the precision matrix on 
cosmological parameter estimation, \citet{dodelson2013,taylor2013,taylor2014} derived a set of requirements 
on the number of independent mock simulations given the number of data points and the number of parameters 
in a given cosmological analysis. Assuming that the estimated covariance matrix follows a Wishart distribution, 
\citep{Sellentin:2016a} present a method for marginalizing over the precision matrix. They demonstrate that the 
resulting likelihood function is no longer Gaussian and instead follows a $t$-distribution.

\citet{dodelson2013,Sellentin:2017a} demonstrate that using a noisy estimate of the precision matrix 
constructed from insufficient number of simulated mocks can systematically bias the confidence 
intervals over the cosmological parameters. These findings place strict requirement on the number of 
mock catalogs for the ongoing and future galaxy surveys. In general, the number of mocks need to be much larger than 
the number of data points.

A common practice in analysis of the deep imaging surveys is tomographic binning of the data and combining probes.
In tomographic binning, the dataset is split into multiple redshift slices and the summary statistics of 
the data are estimated by correlating the data within each redshift slice and cross-correlating the data across 
multiple datasets. The size of the resulting data vector can be as large as a few hundred which makes the task of 
creating mock catalogs even more challenging. 

Combining probes means joint analysis of different cosmic probes such as galaxy 
positions and galaxy shears. Multi-probe analyses are more robust against systematic 
uncertainties and have more constraining power. Most modern surveys offer this opportunity. 
But one of the main obstacles a head of these efforts is the large data vector under consideration. 
Production of the covariance of this data vector requires a very large number of simulations which may not 
be feasible.   

Analytical covariance matrices on the other hand, are not as computationally demanding. Their inverse (precision matrix) 
is noise free. But the model approximations and assumptions used in these methods may not be sufficient 
for the accuracy requirements of precision cosmology. Analytical covariance matrices are considered to 
receive a Gaussian contribution and a non-Gaussian contribution. Assuming that the density field is a 
Gaussian random field, the Gaussian part of the covariance matrix is straightforward to compute \citep{grieb2016,klaus2016,slepian2016b}.

Including the non-Gaussian part of the covariance matrix however, is necessary for accurate parameter estimation \citep{takahashi2011,blot2016,chan2016}.
Significant progress has been made in modeling the non-Gaussian contribution to the covariance matrix through employing approaches based on perturbation 
theory and effective theories of large scale structure \citep{mohammed_seljak,mohammed2017} or by adopting a halo model \citep{takada_spergel,eifler2014}.
Regardless of the applicability of analytical covariance matrices in analyzing the survey data, production of simulated mocks remains an important step 
in \emph{validation} and \emph{calibration} of these analytical covariance matrices \citep{slepian2016b,hildebrandt2017}.

Progress has been made in developing methods that take into account our prior beliefs 
about the of covariance matrices. By considering sparsity of the covariance matrix, \citet{paz2015,padmanabhan2016} 
present methods to estimate the precision matrices with fewer mocks. Shrinkage methods \citep{pope2008,joachimi2016} 
also require less simulations. \citet{fried2017} presents a method for expansion of the precision matrix 
around an analytical model and estimating the leading expansion terms with fewer simulations.

One of the common tools for estimation of covariance matrices is internal resampling of the data such as jackknife and bootstrap. These covariance 
estimators have been extensively used in analyses of galaxy clustering and galaxy-galaxy lensing (for example see \citealt{reid2014,hod_vs_sham,singh2016,shirasaki,kwan2017}).
In situations when no reliable analytical or simulation-based estimate of covariance is available, one could use these estimators. 
For instance in interpreting the SDSS DR7 galaxy clustering measurements, we make use of estimates of covariance matrix based on 
jackknife resampling of the data. But as \citet{norberg,fried2016} point out, data re-sampling methods do not faithfully capture the 
uncertainties. 

Precision estimation of the covariance matrix from mocks requires a large number of mocks to be used in a given analysis. This has 
led to development of a number of \emph{approximate} \emph{methods} for mass production of galaxy and halo (virialized regions of dark matter 
over-density) mock catalogs. These methods rely on approximate gravity solvers for generation of the dark matter density field and a statistical 
prescription for populating the dark matter density field with mock galaxies (see for example \citealt{manera2013,qpm,eazymock,kitaura2016} and references 
therein). The state-of-the-art method for production of mock catalogs are QPM \citep{qpm} and PATCHY \citep{kitaura2016}. These methods were used in 
cosmological analyses of the SDSS III-BOSS final data release \citep{alam2016}.

In the PATCHY method the density field is generated with perturbation theory. The expected number of galaxies in a cosmic volume given the 
density field is computed according to a biasing relation whose parameters are unknown. 
Finally, the number of galaxies is drawn from a negative binomial distribution which is a Poisson distribution with an additional stochasticity. 
\citet{nifty} demonstrates that this approach is very powerful at reproducing the two-point and three-point statistics of galaxies with high accuracy towards mildly 
nonlinear scales.

In Chapter 4, we address two limitations of this method: \emph{bias} \emph{estimation} and \emph{accuracy on small scales}. We replace the pertubation theory-based 
gravity solver with the fast particle mesh (PM) method of \citet{fastpm}. The main difference between the two PM implementations of \citet{fastpm} and \citet{qpm} is that 
FastPM imposes exact large scale growth by applying corrections and does not suffer from the loss of power on large scales as a result of using a limited number of time 
steps for solving equations of motions. We also introduce an automatic bias estimation method based on MCMC. 

In comparison with the previous version of the code based on perturbation theory, the novel approach yields a higher accuracy towards nonlinear scales for both two-point 
and three-point statistics. Additionally, the effective bias model bypasses the resolution requirements of. In order to reach reasonable accuracy, the common approximate $N$-body methods \citep{qpm,fastpm,ice_cola} require a larger grid size than the number of particles. That is, the gravitational forces between the particles need to be calculated on a mesh (at least) twice larger than the number of particles (see also \citealt{nifty,monaco2016} for more details). We note that our effective bias prescription for populating the dark matter density field with halos bypasses this requirement and therefore significantly reduces the computational cost of generating mocks.      
  
Gravitational lensing of luminous sources (galaxies, CMB) can measure 
the state of the inhomogeneous matter density and the time-dependence of dark energy. 
Lensing of the light emitted by background faint galaxies by the intervening large scale structure 
results in very small but correlated distortions in the shapes of galaxies. This phenomenon is referred to 
as cosmic shear.

Cosmic shear studies are hurdled by a number of statistical uncertainties (unknown distribution 
of galaxy intrinsic morphologies) and systematics. The long list of systematic error budget 
of cosmic shear studies can in general fall into the following categories: uncertain distance 
(photometric redshift) information, and imperfect knowledge of intrinsic alignments of galaxies and non-linear matter 
power spectrum models, and image systematics. 

Tomographic reconstruction of the distribution of dark matter with imaging surveys requires 
accurate distance information which is encoded in the redshifts of galaxies. 
The redshift estimated from the imaging surveys is usually uncertain as it is measured with coarse 
spectra in the form of a few magnitude numbers based on broadband photometry. These redshifts are called 
photometric redshifts and their accuracy depends on the number of broadband filters and the overlapping spectroscopic 
sample of galaxies \citep{bonnett2016,choi2016,boris2016,hildebrandt2017}. %FIXME BORIS AND HOGG?

Intrinsic alignment of galaxies (IAs) also contaminate the cosmic shear signal. These IAs, if not accounted for, 
can significantly bias the cosmological parameter inference \citep{codis2015,joachimi2015, kirk2015,krause_ia}. 
Cosmological interpretation of the cosmic shear signal requires an accurate knowledge of the nonlinear 
matter power spectrum \citep{semboloni2013, eifler2015, schaye2015, joudaki2016, kitching2016, mead2016}. Modeling the role of 
Baryonic feedback and the accuracy of the emulators for nonlinear matter clustering remains an active area of research. 

Perhaps the most widely and extensively studied sources of systematics in weak lensing are the image systematics which make 
the process of inferring the cosmic shear from noisy images of galaxies far from trivial. The light profile of 
galaxies is distorted with the atmosphere and telescope optics. The amount of distortion is captured in a kernel called 
the Point Spread Function (PSF). The size and anisotropy of the background galaxies used in weak lensing studies is 
smaller than those of the PSF, making the cosmic shear signal dominated by the PSF which is empirically estimated at 
the positions of stars and then interpolated to the positions of galaxies.  

Furthermore, additional systematics arise from unrealistic models of galaxy light profile \citep{voight2010,im3shape,kac2014}, 
detector effects \citep{arun2016,jaya2016,plazas2016}, nonlinear dependence of galaxy ellipticities 
on the pixel data in the presence of noise that biases the shear \emph{point} \emph{estimates} \citep{melchoir,great3,conti2017}, 
and incorrect model of the PSF or inaccurate PSF interpolation \citep{rowe2010,kuijken2015,great3,des}.

These systematics lead to shear calibration biases in the form of additive and multiplicative 
shear biases. For instance, an incorrect size or ellipticity in the PSF model could introduce 
a multiplicative bias shear. This bias gives rise to a spurious correlation function with the 
same order of magnitude as the cosmological signal.

The shear biases are either calibrated with realistic image simulations \citep{im3shape,jee2016,conti2017} or through cross correlation of 
the cosmic shear signal with other cosmic probes that do not suffer from the same biases (\eg\ galaxy positions)
\citep{liu2016,schann,singh2017}. 

Simulation based techniques are limited due to difficulty in 
generation of realistic galaxy image simulations \citep{great3,lanus2017} and our limited ability to match the depth and detection limits of 
the imaging surveys \citep{hoekstra}. The limitation of cross-correlation technique is that it makes strong assumptions about the assigning a 
single bias parameter to a large population of galaxies under consideration. The shear biases have been shown to be scale-dependent \citep{des,jee2016} 
and depend on galaxy properties \citep{im3shape,conti2017}. 

Novel shear inference models that do not rely on delivering a point-estimate for the shear signal \citep{schneider,bernstein,huff,sheldon} 
provide promising venues for mitigation of biases but their accuracy is limited to the accuracy in the PSF model.
In chapters 1 and 5, we try to address two challenges in accurate PSF modeling that could lead to biases in cosmological weak lensing analysis. 

One of the steps of astronomical image processing that could lead to inaccurate model of the PSF is centroiding 
of stars. An erroneous method for determination of the centroid of stars could lead to both inaccurate estimation of the 
PSF at the positions of stars and inaccurate PSF interpolation \citep{anderson2000,lupton,anderson2003,desc}. Large astronomical surveys rely on approximate 
methods for centroiding the stars \citep{sextractor,sdss,des}. 

In chapter one we argue that in the presence of noise, there exist
a lower bound on the error arising from centroiding methods. We show that this theoretically set lower bound, also known as the Cram\'{e}r-Rao lower bound \citep{lecam}, 
is \emph{almost} saturated by centroiding methods that rely on correlating the images of stars with the PSF or some approximation to the 
PSF. These centroiding methods are called matched-filter centroiding methods and in certain limits provide an approximation to the more accurate 
methods that rely on fitting a model to the stars. 

Additionally, weak lensing shear couples the short wavelength and long wavelength modes of the galaxy light profiles. 
Therefore estimation of the response of galaxy light profile to shear requires knowing the model of galaxy image 
with a resolution better than the PSF that the image of galaxy is convolved with. 

In most space telescope \citep{euclid,wfirst} the detectors are designed to be large in order to yield wider field of view. 
That is, the detectors sample the light profile of the PSF at a lower resolution. In such cases, a large fraction of the light of a given point source 
is captured by the pixel that contains the centroid of the point source. In other words, the PSF is \emph{undersampled}.

Most galaxies of interest for weak lensing studies with these space-based weak lensing experiments are unresolved. 
As a consequence, estimating the shear from unresolved galaxy images requires 
knowing the the galaxy light profile model and the PSF at higher resolutions than the native pixel size of the telescopes \citep{olic,ngole2}.

The filters in HST WFC3 IR channel have the most undersampled detectors among all HST filters. 
As such, HST WFC3 IR channel could benefit a lot from an accurate model of its super resolution PSF. 
In this chapter 5, we present the first generative forward model of stellar sources as observed by the HST camera.
Unlike previous attempts at inferring the PSF model of HST WFC3, our model is data driven in that it makes 
use of the observed data and it presents a proper treatment of modeling in the presence of damaged pixels, overlapping 
point sources, \etc\.  

Galaxies are luminous tracers of the large scale structure. Thus study of galaxy clustering as a means to understand the growth of structure is 
one of the key drivers of low redshift galaxy surveys. Significant progress has been made in linking the observations of 
galaxies and our theoretical understanding of the nonlinear evolution of dark matter. 

On large scales, cosmological perturbation theory equipped with a prescription for galaxy bias permits us 
to accurately model the clustering of galaxies. However, nonlinear evolution of matter 
poses a challenge to applicability of perturbative approaches in small scales. In such limits, 
galaxy clustering must be understood within the context of the halo model which has had 
much success in describing time evolution of the clustering of matter.  

Modeling galaxy clustering with the halo model follows two central assumptions. First, galaxies form in virialized regions of 
dark matter over density known as halos. Clustering of galaxies is governed by clustering of dark matter halos. 
In order to use galaxy clustering measurements for gaining insight into cosmological structure formation models we need to 
know how clustering of galaxies can be determined from clustering of halos. That is, we are required to specify how halos 
are populated with galaxies \citep{seljak2000,scoccimarro2001,berlind_weinberg2002}.

One of the successful prescriptions for assignment of galaxies to halos is the halo occupation distribution (HOD). 
HOD provides a prescription for the expected number of galaxies that reside a halo as well as 
the positions and velocities of galaxies that are distributed within a halo. This simple assumption has been useful 
in reproducing the observed statistics of galaxies \citep{tinker_rsd2007,zehavi2011,zheng_guo}.

Despite being successful at reproducing a wide range of the observed statistics of galaxies, the simple \emph{mass-only} HOD 
remains challenged by a theoretical phenomenon known as assembly bias. Assembly bias, seen in $N-$body simulations, states that at a fixed halo mass, clustering 
of halos depend on halo properties beyond mass such as their formation history, gravitational potential, \etc\ \citep{weschler2006,gao2007,arz2014,sunayama2016}.
This theoretical prediction has not yet been confirmed with observations and there are very mixed results in the literature.

By weak lensing analysis of the galaxy groups in GAMA (GAlaxy and Mass Assembly, see \citealt{driver}) 
with the KiDS (Kilo Degree Survey, see \citealt{kuijken2015}) imaging data, \citet{dvornik2017} find no evidence for halo assembly bias.
Using the galaxy-galaxy lensing and clustering measurements of SDSS galaxy \emph{redmapper} clusters \citep{rykoff} with halo masses $10^{14}$ $M_{\odot}$, 
\citet{miyatake2016} claimed strong difference between the bias of two populations of clusters with different radial distribution of satellite galaxies 
(a proxy for the formation history of a dark matter halo). \citet{zuetal2016} argues that the findings of \citet{miyatake2016} is due to projection effects (as a result of highly uncertain photometric redshifts) and they cannot be interpreted as detection of assembly bias. 

The effect of assembly bias on galaxy clustering can be seen in many subhalo abundance matching (SHAM, see \citealt{hearin2014,lehman2015} and references therein) methods. SHAM assumes a one-to-one relation between halo (including subhalos) properties (\eg\ mass, circular velocity) and some galaxy properties (\eg\ stellar mass, luminosity). In Chapter 3, with the clustering measurements of SDSS DR7 main sample of galaxies \citep{abazajian2009} and the Small MultiDark $N$-body simulation, we show that the clustering predictions of \citep{decorated} HOD model that takes assembly bias into account are consistent with the predictions of the \emph{mass-only} HOD model within 1-$\sigma$ level. 

We also note that for the sample of $L_{\star}$ galaxies, there is \emph{slight} improvements in galaxy clustering predictions of the HOD model with assembly bias 
on large scales. Furthermore, we note that in terms of information criteria, there is no improvement when a more complex HOD model is used to fit the clustering measurements.
We note that in terms of the effect of halo assembly bias on galaxy clustering, our constraints show qualitatively similar behavior in comparison with the predictions of subhalo abundance matching models. Unlike the findings of \citet{zentner2016}, we do not find any evidence for assembly bias in the satellite population of galaxies.

One of the perplexities faced by contemporary cosmology is the discordance between the constraints on cosmological parameters from some of the low redshift probes and 
the Planck CMB results \citep{planckII}. This includes the cosmological constraints from the cosmic shear analysis of CFHTLenS \citep{heymans,kitching2016} and KiDs \citep{hildebrandt2017}. The disagreement appears in the constraints over the parameters $\sigma_{8}$ and $\Omega_{m}$, the amplitude of the linear matter power spectrum and the matter density parameters respectively. This tension exists at a 2-$\sigma$ level. 

The \emph{cosmic} \emph{discordance} persists when the low redshift probe under consideration is galaxy-galaxy lensing which contains information regarding galaxy-matter cross correlation. Similar to galaxy clustering, modeling the small scale galaxy-galaxy lensing signal requires assuming a halo model. 
This signal is an estimate of the lensing of the background galaxies from a \emph{deep} imaging survey and the foreground galaxies in a \emph{shallow} spectroscopic survey.

The galaxy-galaxy lensing measurements of SDSS III/BOSS \citep{miyatake15,lensingislow} are not consistent with the predictions of best-fit Planck cosmology \citep{planckII}.
The $\sigma_{8}-\Omega_{m}$ constraint from clustering and galaxy-galaxy lensing measurements of BOSS galaxies \citep{more15} does not match the Planck constraint.
Disagreements hold for the cosmological constraints---assuming a simple mass only HOD model---from the clustering and galaxy-galaxy lensing measurements of SDSS DR7 main sample \citep{cacciato13}. These discrepancies could signal a new physics, observational systematics, or uncertainties arising from the galaxy formation physics, including \emph{assembly} \emph{bias}.

Combining the early universe probes of cosmology and large scale structure requires accurate 
characterization of systematics and nuisance parameters in both datasets. In the context of large scale 
structure, models governing the galaxy-halo connection serve as nuisance parameters that we marginalize over. A possible source of uncertainty in characterization of galaxy-halo connection is assembly bias. In order to distinguish between physical scenarios pointing at potentially new physics and systematics, it is important to study the effect of halo assembly bias on low redshift cosmological probes including galaxy clustering and galaxy-galaxy lensing. In Chapter 3, we take a step toward better understanding the impact of assembly bias. 

The future low redshift probes and the key CMB datasets will enable us to map the 
density field across a considerable fraction of the observable universe. They will also provide us with 
a consistent picture of the initial conditions, formation of structure, and laws of gravity. 
Reaching a percent-level constraint over the cosmological parameters will 
forever shape our understanding of the cosmos. 

Cosmological inferences rely on assuming a specific functional form for the likelihoods. For instance, interpretation of the 
two-point correlation functions of the galaxy shears or galaxy positions is done through assuming a multivariate Gaussian likelihood 
and interpreting the Cluster number counts is done through assuming a Poisson likelihood function. Therefore common practices in 
cosmology make string assumptions about the underlying distribution of the \emph{estimated} observables.

This fundamental assumption may not hold in detail and a robust test of this assumption can only be done with a large 
number of realistic and accurate mock catalogs.One might expect the Gaussianity assumption in large scale structure analyses 
to break down in certain regimes. One of the common arguments for the Gaussianity assumption is the central limit theorem. That is, a sufficiently large copies 
of random variables, Fourier modes of the density field in a cosmological context, are Gaussian distributed. The central limit theorem may not be applicable on 
the very large scales where the number of Fourier modes is not large. Nonlinear evolution of the density field on small scales may also break down this assumption. 

The presence of systematics such as missing data as a result of fiber collision, selection effects, color-magnitude cuts, or catastrophic redshift 
failures \citep{Guo:2012a,Ross:2012aa,Hahn:2017a} may affect the distribution of the estimated summary statistics of the data in a nontrivial way 
that could be difficult to model. Unconventional observables beyond two-point statistics such as the group statistics are not very 
robust against methodological systematics such as group identification (see \citealt{berlind2006,campbell2015}). Modeling such systematics in an explicit likelihood analysis may not be possible. A similar summary statistics in the context of weak lensing is the convergence peak counts. Convergence peak counts can, in principle, deliver competitive cosmological constraints but they suffer from nontrivial impact by systematic uncertainties \citep{abcwl2}. 

Furthermore as we have mentioned earlier, parameter estimation with large scale structure observables require a precise estimate of 
the sample variance in terms of the covariance matrix. Estimation of the covariance matrix is computationally demanding \citep{nifty,harnois} and even 
in the presence of an estimate of it the form of the likelihood function can no longer be characterized by a Multivariate Gaussian distribution \citep{Sellentin:2016a,Sellentin:2017a}.
These problems can lead to biases in cosmological parameter estimation.

On the other hand, parameter estimation can be done with Approximate Bayesian Computation (ABC) without explicitly specifying a likelihood function. 
In ABC, accounting for the likelihood function is implicitly done through a generative simulation based inference and a distant metric that quantitatively 
measures the closeness of the observations and simulations \citep{optimalkernel,abcpmc}. Accounting for systematics in the ABC approach is done by implementing 
them in the generative forward model of the data. For instance, the systematics associated with identification of galaxy groups can be built into the forward model that posses these group mis-identifications. The uncertainty in the photometric redshifts in deep imaging surveys can be incorporated into the forward model of the data.   

This approach also bypasses any assumption regarding the functional form of the underlying distribution of the data. Regardless of how the estimated summary statistics are 
distributed, one could use the ABC method for inferring the parameters of interest.
Dependence of the analysis on the precise knowledge of the precision matrix can be circumvented by accounting for the sample variance 
in the forward model. It is important to note that, the accuracy of this approach relies on the distance metric. Considerable care needs to be devoted to finding and testing a 
reliable distance metric which in general depends on the cosmological observable under consideration \citep{abcsn,abcwl,jennings2016a,jennings2016b}. 

In Chapter 2, we present the first application of ABC in the context of large scale structure. In particular, we show that this approach can be reliably used in order to constrain 
the galaxy-halo connection models with small-scale galaxy clustering and group statistics. Using a simulated data with known true HOD parameters, we show that one can use ABC to infer these true parameters. We show that with ABC, observables beyond conventional two-point statistics such as the abundance of galaxy groups can robustly constrain the galaxy-halo connection parameters. More importantly, we make comparison between the performance of ABC equipped with Particle Monte Carlo sampling and Gaussian Pseudo-likelihood with MCMC sampling. We show that the constraints obtained from the two approaches are comparable and consistent within 1-$\sigma$ level, with the ABC-PMC constraints slightly less 
biased when the abundance of galaxy groups is used as the summary statistics of the data. 

Furthermore, we discuss why the ABC-PMC method is advantageous in terms of implementing the sample variance and observational systematics in the forward model.
We also point out the limitations of applying such technique to cosmological parameter estimation with galaxy clustering measurements of the modern galaxy surveys. 
We argue that the main computational bottleneck is going to be the forward model that includes all observational systematics and complicated structure formation models. 

In this thesis, we try to address different aspects of solving a cosmological problems which is equivalent to specifying the following general likelihood function:

\begin{equation}
\mathcal{L} = p(\mathbf{d} \; | \; \mathbf{\mu} , \mathbf{C}),
\end{equation}

where $\mathbf{d}$ is an estimated data vector, $\mathbf{\mu}$ is a theoretical prediction, and \mathbf{C} is a covariance matrix that characterize the uncertainties in 
a given cosmological analysis. In Chapters 1 and 5, we try to address the systematic uncertainties that affect $\mathbf{d}$ in weak lensing analyses. In Chapter 3 we try to address one of the astrophysical uncertainties, \emph{assembly bias}, that could impact our cosmological parameter inference practices with galaxy clustering observations. In Chapter 4, we address how accurately one could generate simulated datasets needed for precise estimation of the error covariance matrices of galaxy clustering measurements. In Chapter 2, we try to address how we can bypass the need for specifying a functional form for the likelihood function $p(\mathbf{d} \; | \; \mathbf{\mu} , \mathbf{C})$ and the challenges associated with it.

Chapter 1 of this thesis is a joint work with David~W.~Hogg. The underlying ideas and the direction of this project were developed by David~W.~Hogg and me. 
All the codes and text for this chapter was written by me with significant edits and modifications from David~W.~Hogg. Chapter 2 is based on a joint work with ChangHoon~ Hahn, Kilian~Walsh, Andrew~Hearin, David~W.~Hogg, and Duncan~Campbell. ChangHoon~Hahn and I had equal contributions to development of bulk of the code and text for this project. Kilian~Walsh also had significant contributions to both coding and writing. Andrew~Hearin and David~Hogg contributed to writing of the paper and Duncan~Campbell had some contribution 
to the code. All of the code and the text for Chapter 3 was written by me. ChangHoon~Hahn had significant contribution to editing the text. I also received valuable comments from 
David~W.~Hogg, Alex~I.~Malz regarding the text and from Andrew~Hearin and Chia-Hsun~Chuang regarding the highly technical details of my calculations. The idea of Chapter 4 was developed by Francisco-Shu Kitauro, Yu~Feng and myself. I wrote all the code and text, with some contribution to the text from Francisco-Shu~Kitaura. For the computations related to this work, I received significant help from Yu~Feng, Gustavo~Yepes, Cheng~Zhao, Chia-Hsun~Chuang, and ChangHoon~Hahn. The idea of Chapter 5 was developed in collaboration with Ross~Fadely and David~W.~Hogg. Development of the code was done by me and under the supervision of Ross~Fadely and David~W.~Hogg. All the text was written by me with edits by David~W.~Hogg.

All of the code that I have written for the computations in this thesis are publicly available except the PATCHY code which is being ready for a public release. 
The code for Chapters 1,2,3, and 5 are available at \url{github.com/mjvakili/centerer}, \url{github.com/mjvakili/ccppabc}, \url{github.com/mjvakili/gambly}, and \url{github.com/mjvakili/supermean}.

%These ambitious goal however, can only be achieved if we keep the 
%uncertainties arising from astrophysical and observation systematics 
%below the statistical errors. In this dissertation, we attempt to take advantage of 
%computational methods in order to address some sources of uncertainty that stand on the way of 
%reaching precision cosmology. 

%We live in the era of systematic dominated cosmological measurements. 
%Therefore it is important to make use of statistical techniques that mitigate 
%biases arising from these systematic errors. It is important to design end-to-end simulations of the survey.

%Galaxy surveys will provide a valuable legacy dataset that can be used for detailed galaxy formation studies. 
%These investigations will provide a window into how the visible structures in the observable universe 
%trace the underlying invisible dark matter density. Furthermore, more comprehensive knowledge of galaxy-dark matter connection 
%can inform cosmological parameter inferences with galaxies. 

In Chapters 2,3, and 4 we used the suite of MultiDark cosmological $N$-body simulations made publicly available in the CosmoSim database \footnote{\url{https://www.cosmosim.org}}. The CosmoSim database used in this paper is a service by the Leibniz-Institute for Astrophysics Potsdam (AIP). sThe MultiDark database was developed in cooperation with the Spanish MultiDark Consolider Project CSD2009-00064.

In Chapter 3, we used the measurements done with the SDSS DR7 \footnote{\url{http://classic.sdss.org/dr7/}} data \citep{abazajian2009}. 
Funding for the SDSS and SDSS-II has been provided by the Alfred P. Sloan Foundation, the Participating Institutions, the National Science Foundation, the U.S. Department of Energy, the National Aeronautics and Space Administration, the Japanese Monbukagakusho, the Max Planck Society, and the Higher Education Funding Council for England. The SDSS Web Site is http://www.sdss.org/. The SDSS is managed by the Astrophysical Research Consortium for the Participating Institutions. The Participating Institutions are the American Museum of Natural History, Astrophysical Institute Potsdam, University of Basel, University of Cambridge, Case Western Reserve University, University of Chicago, Drexel University, Fermilab, the Institute for Advanced Study, the Japan Participation Group, Johns Hopkins University, the Joint Institute for Nuclear Astrophysics, the Kavli Institute for Particle Astrophysics and Cosmology, the Korean Scientist Group, the Chinese Academy of Sciences (LAMOST), Los Alamos National Laboratory, the Max-Planck-Institute for Astronomy (MPIA), the Max-Planck-Institute for Astrophysics (MPA), New Mexico State University, Ohio State University, University of Pittsburgh, University of Portsmouth, Princeton University, the United States Naval Observatory, and the University of Washington.

In Chapter 5, all of the HST archival data were obtained from the Mikulski Archive for Space Telescopes (MAST) \footnote{\url{https://archive.stsci.edu}}. STScI is operated by the Association of Universities for Research in Astronomy, Inc., under NASA contract NAS5-26555.
 
Bulk of the computations in this work were carried out on the Mercer cluster which is part of the New York University High Performance Computing facilities \footnote{https://wikis.nyu.edu/display/NYUHPC/Clusters+-+Mercer}}. For a fraction of computations performed for this thesis, I acknowledge the use of machines in the Center for Cosmology and Particle Physics \footnote{http://ccpp.nyu.edu}}.
I also acknowledge the use of the computational resources in MareNostrum Cosmological Project \footnote{http://astro.ft.uam.es/marenostrum/}}. 



\chapter*{Introduction}\addcontentsline{toc}{chapter}{Introduction}

Over the past few years, the field of extrasolar planet (exoplanet) research
has really taken off thanks, in large part, to the exquisite time series
photometry measured by the \kepler\ Mission \citep{Borucki:2010}.
The Mission enabled the discovery of thousands of planets and planet
candidates outside the Solar System \citep{Rowe:2015}.
The zoo of planetary systems is extremely diverse---with sizes, masses, and
orbital periods spanning orders of magnitude---and the statistics are now
sufficient to test theories of planet formation and evolution.

The \kepler\ Mission has changed the face of exoplanet research because of its
photometric precision and the sheer volume of the dataset.
In order to discover small planets that serendipitously transit their host
stars, the \kepler\ spacecraft was designed to monitor the brightness of about
150,000 stars in one $10^\circ \times 10^\circ$ patch of the sky nearly
continuously---at a half-hour cadence---for more than three years with a
relative precision of a few parts-per-million for the brightest stars.
The Mission surpassed its fiducial goals and took data for over 4~years before
two of the reaction wheels used to stabilize the pointing failed in the Spring
of 2013.

Despite the fact that most planets never transit their host star---based on
geometric effects alone---and the fact that transit surveys are most sensitive
to large planets on short orbits, the discoveries made in the \kepler\ dataset
and careful characterization of the selection effects and search completeness
have enabled detailed studies of the true underlying distribution of planets
over a wide range in parameter space \citep[examples include][and
\chap{exopop}]{Howard:2012, Petigura:2013, Foreman-Mackey:2014,
Dressing:2015}.
These observational studies of the population of exoplanets are arguably the
ultimate goal of the \kepler\ Mission because they open the door to direct
comparison with theories of planet formation and evolution.

The different constraints on the intrinsic rate and distribution of planets
differ in detail but several overarching results are solid.
The evidence suggests that every cool M-star has at least one planet in orbit
\citep{Dressing:2013, Dressing:2015} and more than half of the other main
sequence stars should have planetary systems \citep{Howard:2012, Fressin:2013,
Petigura:2013, Foreman-Mackey:2014}.
Of these planets, the most intrinsically common are in the ``super-Earth'' or
``mini-Neptune'' range from about twice to four-times the radius of Earth.
A combination of radial velocity follow-up and hierarchical inference methods
indicate that most of these mini-Neptunes gaseous instead of rocky
\citep{Weiss:2014, Rogers:2015} but since there are no planets like this in
the Solar System, understanding them is crucial to our theories of planetary
system formation.

% These results can be used as targets for numerical simulations of planetary
% systems or population level inferences can be combined with theoretical models
% to \citep{Wolfgang:2014, Rogers:2015}.

One shortcoming of the \kepler\ Mission was that it only targeted one field
and in that frame, the main focus was on relatively faint F, G, and K dwarf
stars.
This target selection was chosen to enable the study of long-period planets
and the discovery of Earth-sized planets orbiting Sun-like stars.
Unfortunately many of these stars and their planetary systems are not amenable
to radial velocity follow-up because the star is too faint to achieve the
required velocity precision or the expected velocity amplitude is too small to
detect.
In the Summer of 2014, the \kepler\ instrument was re-purposed and it began
taking data in a mode called \KT\ with substantial degraded pointing accuracy
\citep{Howell:2014}.
Because of technical constraints, \KT\ targets a different field in the
ecliptic plane every three months.
This means that it can target stars in different environments and focus on
gathering the census of planets orbiting bright, nearby stars.
It has been demonstrated that the data from \KT\ can reach precisions
comparable to the original Mission and that it can be used to discover
transiting exoplanets \citep[][and \chap{ketu}]{Vanderburg:2014,
Vanderburg:2015, Crossfield:2015, Foreman-Mackey:2015}.
The discoveries made using \KT---and the upcoming \tess\ Mission---improve our
knowledge of the population of exoplanets, especially those planets that orbit
the cool M-stars that were not prioritized by the \kepler\ Mission.
These discoveries also present excellent targets for radial velocity follow-up
and even spectroscopic observations of their atmospheres using the planned
\project{James Webb Space Telescope}.

The technical problem of searching for transits in the massive datasets
produced by a time series mission like \kepler\ is a hard one.
The relative change in brightness caused by the transit of planet in front of
its host star is given by the area ratio between the planet and star
\citep{Winn:2010}.
Therefore, when an Earth-sized planet transits a Sun-like star, the amplitude
of the signal is smaller than 100 parts-per-million.
What's more, in the case of the Earth's orbit, this signal would only last for
a little over half a day, once every year.
Add to this the fact that most light curves are fraught with signals induced
by stellar variability \citep{Basri:2013}, spot activity
\citep{McQuillan:2014}, and instrumental effects \citep{Stumpe:2012,
Smith:2012} with amplitudes far exceeding most transits.
In order to find transits, we must, therefore, develop methods for efficiently
and robustly mining large sets of light curves for tiny, sparse signals.
Nearly all transit search algorithms rely on some sort of matched filter that
is made insensitive to noise by pre-processing the light curves to remove the
trends or by designing an estimator that is insensitive to these effects
\citep[][]{Kovacs:2002, Kovacs:2005, Berta:2012, Petigura:2013,
Foreman-Mackey:2015}.

Despite the attempts made to develop search algorithms that are robust to
systematics and variability, all automated search results are completely
dominated by false signals induced by poorly characterized noise in the light
curves.
In practice, automatic removal of these events has not been demonstrated to be
sufficient---although the results are starting to look promising
\citep{Jenkins:2014}---and all published catalogs of planet candidates are
manually vetted.
This means that the published list of candidates is \emph{chosen by a
person---or group of people---going through the data by hand}.
This method is not efficient or scalable so a substantial set of heuristic
filtering is applied to the candidate list even before anyone looks at the
light curves.
One of the standard filters is to only consider candidates with at least three
observed transits \citep[for example][]{Petigura:2013, Burke:2014, Rowe:2015}.
This greatly restricts the range of parameter space that can be search.
In particular, these methods will miss any planets on orbits longer than a
fraction of the survey baseline.

While transit surveys present the most effective means for systematic
exoplanet characterization, their use is limited by existing transit search
methodologies for planets with long orbital periods.
In many cases, massive long-period planets dominate the dynamics of the
planetary systems---like Jupiter in the Solar System---but their existence is
completely missed by \kepler.
This shortcoming becomes even more severe for \KT\ and \tess, transit surveys
with shorter baselines.
The final \chapname\ of this dissertation presents a novel method for transit
search designed specifically to discover and quantify these important planets.

The study of exoplanets and their population has been driven by the public
\kepler\ dataset and, in particular, by methods and software solutions
developed by graduate students and young researchers around the world to
squeeze all the available information out of the existing dataset.
This dissertation presents methods developed with exactly these goals in mind.
Each \chapname\ is accompanied by an open source implementation of the method
and code to reproduce the results and figures.
Of these projects, the most popular is the Markov Chain Monte Carlo
implementation \project{emcee} \citep[][and
\chap{emcee}]{Foreman-Mackey:2013}.
With nearly 300 citations at the time of
writing\footnote{\url{http://adsabs.harvard.edu/cgi-bin/nph-ref_query?bibcode=2013PASP..125..306F&amp;refs=CITATIONS&amp;db_key=AST}}
and an active community on
\project{GitHub}\footnote{\url{https://github.com/dfm/emcee}}, \project{emcee}
has enabled many modest and ambitious probabilistic inferences across
astrophysics.

\chapname s~\chapalt{emcee} and~\chapalt{exopop} have both been refereed and
published in the astronomical literature.
\Chap{ketu} has been submitted to \emph{The Astrophysical Journal} and updated
in response to the referee's comments.
\Chap{peerless} is in preparation for submission.
All of these \chapname s were co-authored with collaborators but the majority
of the work and writing in each \chapname\ is mine.
Here, I describe my specific contributions to each \chapname:
\begin{enumerate}

{\item For \chap{emcee}, I generalized the algorithm proposed by
\citet{Goodman:2010} through discussions with Jonathan Goodman and David Hogg.
I implemented the algorithm with contributions from Dustin Lang and wrote the
paper with some additions by David Hogg.}

{\item For \chap{exopop}, I developed the project idea in collaboration with
David Hogg and Timothy Morton.
I then implemented the project and wrote the paper with contributions from
David Hogg.}

{\item Of the published \chapname s, \chap{ketu} was the most collaborative.
I developed the idea for the algorithm building on previous work with David
Hogg, Dun Wang, and Bernhard Sch\"olkopf.
Using this algorithm, I wrote the code to search for transits in the
\project{K2} Campagin 1 dataset and deployed it on the NYU HPC Butinah
cluster\footnote{\url{http://nyuad.nyu.edu/en/research/infrastructure-and-support/nyuad-hpc.html}}.
I wrote the majority of the paper with Sections contributed by Ben Montet and
Timothy Morton.}

{\item The fundamental ideas underlying \Chap{peerless} were developed through
discussions with Bernhard Sch\"olkopf and David Hogg.
The implementation and text are mine.}

\end{enumerate}


\chapter*{Conclusion}\addcontentsline{toc}{chapter}{Conclusion}

In this dissertation, we try to address some of the longstanding computational problems associated with 
interpreting the large scale structure and weak lensing datasets.
We show that in the presence of noise, approximate stellar centroiding methods capture most of the 
information available in Fisher information matrix. Our investigation ensures that the PSF bias arising 
from the use of fast centroiding methods will be minimal. 

We show that likelihood free inference methods such as ABC can be used for robust parameter estimation 
in large scale structure cosmology. We demonstrate that that our ABC-PMC method is capable of deliver unbiased 
parameter estimation by incorporating sample variance in the generative forward model. Furthermore, this method can be used 
for inference with summary statistics beyond two point correlation functions such as the group multiplicity function.

We constrain the impact of halo assembly bias on galaxy clustering measurements of the local universe. 
We show that by taking into account the dark matter halo properties beyond mass can lead to slight improvements in 
accuracy of the galaxy clustering predictions on large scales. But we find that the effect is only marginal and is fully consistent 
with the galaxy clustering predictions of a simple mass-only HOD prescription.

We present a novel method for estimation of the galaxy clustering uncertainties in the form of covariance matrices. 
With an accurate N-body simulation, we show that our method is able to model the nonlinear halo two-point and three-point statistics with a 
percent level accuracy needed for the next generation of redshift space distortion and Baryonic acoustic oscillation 
studies with the upcoming galaxy surveys.

Then we present a probabilistic model for estimating the super-resolution PSF of the HST WFC3-IR channel.
We show that our generative forward model can accurately model the point sources observed by this telescope. We are able 
to estimate the PSF at a resolution higher than the native pixel grid. We are also able to robustly model the images of 
point sources in crowded fields in the HST WFC3-IR observations.











\chapter*{Conclusion}\addcontentsline{toc}{chapter}{Conclusion}

In this dissertation, we address some of the longstanding problems associated with large scale structure studies.
We show that in the presence of noise, approximate stellar centroiding methods capture most of the 
information available in Fisher information matrix. Our investigation ensures that the PSF bias arising 
from the use of fast centroiding methods will be minimal. 

The future low redshift probes and the key CMB datasets will enable us to map the 
density field across a considerable fraction of the observable universe. They will also provide us with 
a consistent picture of the initial conditions, formation of structure, and laws of gravity. 
Reaching a percent-level constraint over the cosmological parameters will 
forever shape our understanding of the cosmos. 

Then we present a probabilistic model for estimating the super-resolution PSF of the HST WFC3-IR channel.
We show that our generative forward model can accurately model the point sources observed by this telescope. We are able 
to estimate the PSF at a resolution higher than the native pixel grid.

We show that likelihood free inference methods such as ABC can be used for robust parameter estimation 
in large scale structure cosmology. With a simulation, we show that our ABC-PMC method is capable of deliver unbiased 
parameter estimation by incorporating sample variance in the generative forward model. Furthermore, this method can be used 
for inference with summary statistics beyond two point correlation functions such as the group multiplicity function.

We constrain the impact of halo assembly bias on galaxy clustering measurements of the local universe. 
We show that by taking into account the dark matter halo properties beyond mass, we can accurately model 
the galaxy clustering on large scales. We find that the efect is only present in the central galaxy populations. 

We present a novel method for estimation of the galaxy clustering uncertainties in the form of covariance matrices. 
With an accurate N-body simulation, we show that our method is able to model the nonlinear halo clustering with a 
percent level accuracy needed for the next generation of redshift space distortion and baryonic acoustic oscillation 
studies with the upcoming galaxy surveys.












